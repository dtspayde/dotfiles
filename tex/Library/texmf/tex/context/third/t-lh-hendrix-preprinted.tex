\protect

\setuppapersize[letter][letter] % or [letter][letter]; in that case, you'll have to fidget with the dimensions

\setuplayout[header=0.25in, 
             topspace=1in, % top margin
             backspace=1in, % left margin
             height=9.5in, % height=header+text+footer
             footer=0.75in,
             width=6.5in] % width of text area

\enableregime[utf] % or whatever you prefer

\setupcolors[state=start] % in case you want any colored output

\mainlanguage[en] % or whatever you prefer

\starttypescript [Hendrix]
	\definetypeface [MyHendrix][rm][Xserif][Baskerville]
	% \definetypeface [MyHendrix][rm][Xserif][Garamond]
	\definetypeface [MyHendrix][ss][Xsans][Trebuchet MS]
\stoptypescript

\usetypescript[Hendrix]

\setupbodyfont[MyHendrix,12pt] % this is the small bodyfont for the letterhead; we'll change to a bigger size for the actual letter later
%\setupbodyfont[modern,12pt] % this is the small bodyfont for the letterhead; we'll change to a bigger size for the actual letter later

\setuppagenumbering[state=stop] 

%	\blank[line]%
%	PHONE:  501-450-1251  FAX:  501-450-3829 \\%
%	EMAIL:  spayde@hendrix.edu  WEB:  www.hendrix.edu][]

\setupexternalfigures[directory={/Users/spayde/Library/texmf/tex/context/third, /Users/spayde/Library/texmf/tex/graphics}]
\useexternalfigure [hendrix_logo][HENDRIX_LOGO_158] [scale=510]
\useexternalfigure [signature][spayde_signature][height=0.5in]

\def\HendrixHeader%
	{\cbox
		{\setupframed[align=middle, frame=off]%
		 \vbox to 0.375in{}%
		 \framed{
		\externalfigure[hendrix_logo] \\ 
%		 {\switchtobodyfont[10pt, sans] \kap{c o l l e g e}} \\
		{\switchtobodyfont[10pt, sans, MyMSTrebuchet] C O L L E G E} \\
		{\switchtobodyfont[9pt]
		\cbox{
		\blank[line]
		1600 Washington Avenue \\
		Conway, Arkansas 72032-3080
		}
		}
		}
		}
	}

\def\HendrixFooter%
	{\cbox{\switchtobodyfont[9pt]
		{\switchtobodyfont[8pt, sans, MyMSTrebuchet] DEPARTMENT OF PHYSICS} \\
		\blank[line]
		{\switchtobodyfont[8pt, sans] PHONE:} 501-450-1251 \emspace
%		{\switchtobodyfont[8pt] \ss PHONE:} 501-450-1251
		{\switchtobodyfont[8pt, sans] FAX:} 501-450-3829 \\
		{\switchtobodyfont[8pt, sans] EMAIL:} spayde@hendrix.edu
\emspace
		{\switchtobodyfont[8pt, sans] WEB:} www.hendrix.edu
%	 	FAX: 501-450-3829  \\
%		EMAIL: spayde@hendrix.edu 	  WEB: www.hendrix.edu
	}
	}


\definelogo
	[hendrix_header][top][middle]
	[command=\HendrixHeader]

\definelogo
	[hendrix_footer][footer][middle]
	[command=\HendrixFooter, state=start]

%\setupfooter[text][state=normal, style=normal]

%\setupfootertexts[\placelogos[hendrix_footer]]

\long\def\date{\vbox to 0.7in{\strut}\placelogos[hendrix_footer]\switchtobodyfont[12pt]\setupwhitespace[none]\currentdate[day, month, year]}

\long\def\addressee#1{\blank[line]#1} % Defines field for receiver's address and places it at right position.
%\long\def\addressee#1{\strut\blank[23*medium]\switchtobodyfont[11pt]\setupwhitespace[none]#1} % Defines field for receiver's address and places it at right position.

\long\def\lettersubject#1{\blank[line]{\bf Re: #1}} % Subject line, in bold face.

\long\def\letteropening#1{\blank[line]#1} % opening\ldots

\long\def\letterbody#1{\blank[line]\setupwhitespace[medium]#1} % the main body

%\long\def\letterclosing#1{\blank[line]#1\blank[2*line]\externalfigure[signature]Damon T. Spayde \crlf Assistant Professor of Physics} % and the closing formula
%\long\def\letterclosing#1{\blank[line]#1\blank[2*line]\externalfigure[signature]Damon T. Spayde \crlf Dr. Brad P. Baltz and Rev. William B. Smith Odyssey Associate Professor of Physics\crlf Chair, Department of Physics} % and the closing formula
% \long\def\letterclosing#1{\blank[line]#1 \crlf \externalfigure[signature] \crlf Damon T. Spayde \crlf Dr. Brad P. Baltz and Rev. William B. Smith Odyssey Associate Professor of Physics\crlf Chair, Department of Physics} % and the closing formula
\long\def\letterclosing#1{\blank[line]#1 \crlf \externalfigure[signature] \crlf Damon T. Spayde \crlf Associate Professor of Physics} % and the closing formula

%\setupalign[right] % I prefer letters with a ragged right, but you don't have to\ldots

\unprotect
